% --------------------------------------------------------------
% This is all preamble stuff that you don't have to worry about.
% Head down to where it says "Start here"
% --------------------------------------------------------------


%% \copyright 2022-2023 William Emerison Six

%% Permission is hereby granted, free of charge, to any person obtaining a copy of this software and associated documentation files (the "Software"), to deal in the Software without restriction, including without limitation the rights to use, copy, modify, merge, publish, distribute, sublicense, and/or sell copies of the Software, and to permit persons to whom the Software is furnished to do so, subject to the following conditions:

%% The above copyright notice and this permission notice shall be included in all copies or substantial portions of the Software.

%% THE SOFTWARE IS PROVIDED "AS IS", WITHOUT WARRANTY OF ANY KIND, EXPRESS OR IMPLIED, INCLUDING BUT NOT LIMITED TO THE WARRANTIES OF MERCHANTABILITY, FITNESS FOR A PARTICULAR PURPOSE AND NONINFRINGEMENT. IN NO EVENT SHALL THE AUTHORS OR COPYRIGHT HOLDERS BE LIABLE FOR ANY CLAIM, DAMAGES OR OTHER LIABILITY, WHETHER IN AN ACTION OF CONTRACT, TORT OR OTHERWISE, ARISING FROM, OUT OF OR IN CONNECTION WITH THE SOFTWARE OR THE USE OR OTHER DEALINGS IN THE SOFTWARE.


\documentclass[12pt]{article}

\usepackage[margin=1in]{geometry}
\usepackage{amsmath,amsthm,amssymb,scrextend,commath}
\usepackage{fancyhdr}
\pagestyle{fancy}


\newcommand{\N}{\mathbb{N}}
\newcommand{\Z}{\mathbb{Z}}
\newcommand{\I}{\mathbb{I}}
\newcommand{\R}{\mathbb{R}}
\newcommand{\Q}{\mathbb{Q}}
\renewcommand{\qed}{\hfill$\blacksquare$}
\let\newproof\proof
\renewenvironment{proof}{\begin{addmargin}[1em]{0em}\begin{newproof}}{\end{newproof}\end{addmargin}\qed}
% \newcommand{\expl}[1]{\text{\hfill[#1]}$}

\newenvironment{theorem}[2][Theorem]{\begin{trivlist}
\item[\hskip \labelsep {\bfseries #1}\hskip \labelsep {\bfseries #2.}]}{\end{trivlist}}
\newenvironment{lemma}[2][Lemma]{\begin{trivlist}
\item[\hskip \labelsep {\bfseries #1}\hskip \labelsep {\bfseries #2.}]}{\end{trivlist}}
\newenvironment{problem}[2][Problem]{\begin{trivlist}
\item[\hskip \labelsep {\bfseries #1}\hskip \labelsep {\bfseries #2.}]}{\end{trivlist}}
\newenvironment{exercise}[2][Exercise]{\begin{trivlist}
\item[\hskip \labelsep {\bfseries #1}\hskip \labelsep {\bfseries #2.}]}{\end{trivlist}}
\newenvironment{reflection}[2][Reflection]{\begin{trivlist}
\item[\hskip \labelsep {\bfseries #1}\hskip \labelsep {\bfseries #2.}]}{\end{trivlist}}
\newenvironment{proposition}[2][Proposition]{\begin{trivlist}
\item[\hskip \labelsep {\bfseries #1}\hskip \labelsep {\bfseries #2.}]}{\end{trivlist}}
\newenvironment{corollary}[2][Corollary]{\begin{trivlist}
\item[\hskip \labelsep {\bfseries #1}\hskip \labelsep {\bfseries #2.}]}{\end{trivlist}}

\begin{document}

% --------------------------------------------------------------
%                         Start here
% --------------------------------------------------------------

\lhead{Derivation Of the Cross Product}
\chead{William Emerison Six}
\rhead{\today}

% \maketitle

\copyright 2022-2023 William Emerison Six, licensed under the MIT License (at end of this document)

\begin{problem}{1} %You can use theorem, proposition, exercise, or reflection here.  Modify x.yz to be whatever number you are proving

  Given vectors $\vec{a}$ and $\vec{b}$ in $\R^3$, find a vector in $\R^3$ that is perpendicular both to $\vec{a}$ and to $\vec{b}$.
\end{problem}



\begin{proof}

\textbf{Rotate $\vec{a}$ and $\vec{b}$ so that $\vec{a}$ is on the x axis: }

  Let $\vec{a} \triangleq \begin{bmatrix}
    a_1 \\
    a_2 \\
    a_3 \\
  \end{bmatrix}$ and let $\vec{b} \triangleq \begin{bmatrix}
    b_1 \\
    b_2 \\
    b_3 \\
  \end{bmatrix}$.

  let $k_1 \triangleq \sqrt{a_1^2 + a_2^2}$.

  Define $\vec{f_1}(\vec{v})$ such that $\vec{f_1}(\vec{a})$ is rotated onto the $xz$ plane.

\begin{flalign}
\vec{f_1}(\begin{bmatrix}
     {v_1} \\
     {v_2} \\
     {v_3} \\
\end{bmatrix}
) & \triangleq     \begin{bmatrix}
     \frac{k_1}{\norm{a}} \\
     0 \\
     \frac{-a_3}{\norm{a}} \\
    \end{bmatrix}
* {v_1} +
    \begin{bmatrix}
     0   \\
     1  \\
     0 \\
    \end{bmatrix} * {v_2} +
    \begin{bmatrix}
     \frac{a_3}{\norm{a}} \\
     0 \\
     \frac{k_1}{\norm{a}}\\
    \end{bmatrix}  * {v_3}
\end{flalign}


  Define $\vec{f_2}(\vec{v})$ such that $(\vec{f_2} \circ \vec{f_1})(\vec{a})$ is rotated onto the $x$ axis.

\begin{flalign}
\vec{f_2}(\vec{v}) & \triangleq \begin{bmatrix}
     \frac{a_1}{k_1} & \frac{a_2}{k_1} & 0 \\
     \frac{-a_2}{k_1} & \frac{a_1}{k_1} & 0 \\
     0 & 0 & 1 \\
\end{bmatrix} * \vec{v}
\end{flalign}


\begin{flalign}
(\vec{f_2} \circ \vec{f_1})(\vec{v}) & = \begin{bmatrix}
     \frac{k_1}{\norm{a}} & 0  & \frac{a_3}{\norm{a}} \\
     0 & 1  & 0 \\
     \frac{-a_3}{\norm{a}} & 0 & \frac{k_1}{\norm{a}}\\
    \end{bmatrix} * (\begin{bmatrix}
     \frac{a_1}{k_1} & \frac{a_2}{k_1} & 0 \\
     \frac{-a_2}{k_1} & \frac{a_1}{k_1} & 0 \\
     0 & 0 & 1 \\
\end{bmatrix} * \vec{v}) \\
  & = \begin{bmatrix}
     \frac{a_1}{\norm{a}} & \frac{a_2}{\norm{a}} & \frac{a_3}{\norm{a}} \\
     \frac{-a_2}{k_1} & \frac{a_1}{k_1} & 0 \\
     \frac{-a_1a_3}{k_1\norm{a}} & \frac{-a_2a_3}{k_1\norm{a}} & \frac{k_1}{\norm{a}} \\
\end{bmatrix} * \vec{v}
\end{flalign}

\begin{flalign}
(\vec{f_2} \circ \vec{f_1})(\vec{a}) & = \begin{bmatrix}
     \frac{a_1^2}{\norm{a}} + \frac{a_2^2}{\norm{a}} + \frac{a_3^2}{\norm{a}}  \\
     \frac{-a_2a_1}{k_1}  + \frac{a_1a_2}{k_1} \\
     \frac{-a_1^2a_3}{k_1\norm{a}} + \frac{-a_2^2a_3}{k_1\norm{a}} + \frac{k_1a_3}{\norm{a}} \\
\end{bmatrix} \\
  & = \begin{bmatrix}
     \frac{a_1^2 + a_2^2 + a_3^2}{\norm{a}} \\
     0 \\
     \frac{1}{\norm{a}} * (\frac{-a_1^2a_3}{k_1} + \frac{-a_2^2a_3}{k_1} + {k_1a_3}) \\
\end{bmatrix} \\
  & = \begin{bmatrix}
     \frac{\norm{a}^2}{\norm{a}} \\
     0 \\
     \frac{1}{\norm{a}} * (\frac{-a_1^2a_3}{k_1} + \frac{-a_2^2a_3}{k_1} + \frac{k_1^2a_3}{k_1}) \\
\end{bmatrix} \\
  & = \begin{bmatrix}
     \norm{a} \\
     0 \\
     \frac{1}{\norm{a}} * (\frac{-a_1^2a_3}{k_1} + \frac{-a_2^2a_3}{k_1} + \frac{((a_1^2 + a_2^2))a_3}{k_1}) \\
  \end{bmatrix}  \\
  & = \begin{bmatrix}
     \norm{a} \\
     0 \\
     0
\end{bmatrix}
\end{flalign}


\begin{flalign}
(\vec{f_2} \circ \vec{f_1})(\vec{b}) = \begin{bmatrix}
     \frac{a_1b_1}{\norm{a}} + \frac{a_2b_2}{\norm{a}} + \frac{a_3b_3}{\norm{a}} \\
     \frac{-a_2b_1}{k_1}  + \frac{a_1b_2}{k_1} \\
     \frac{-a_1a_3b_1}{k_1\norm{a}} + \frac{-a_2a_3b_2}{k_1\norm{a}} + \frac{k_1b_3}{\norm{a}} \\
\end{bmatrix}
\end{flalign}


\textbf{Rotate $\vec{b''} \triangleq (\vec{f_2} \circ \vec{f_1})(\vec{b}) $ onto the $xy$ plane }


  let $k_2 \triangleq \sqrt{{b''_2}^2 + {b''_3}^2}$.


  \begin{flalign}
            \vec{b''}  \triangleq \begin{bmatrix}
    b''_1 \\
    b''_2 \\
    b''_3 \\
            \end{bmatrix}   \triangleq
   \begin{bmatrix}
     \frac{a_1b_1}{\norm{a}} + \frac{a_2b_2}{\norm{a}} + \frac{a_3b_3}{\norm{a}} \\
     \frac{-a_2b_1}{k_1}  + \frac{a_1b_2}{k_1} \\
     \frac{-a_1a_3b_1}{k_1\norm{a}} + \frac{-a_2a_3b_2}{k_1\norm{a}} + \frac{k_1b_3}{\norm{a}} \\
\end{bmatrix}
\end{flalign}


\begin{flalign}
\vec{f_3}(\vec{v}) & \triangleq \begin{bmatrix}
     1 & 0 & 0 \\
     0 & \frac{b''_2}{k_2} & \frac{b''_3}{k_2} \\
     0 & \frac{-b''_3}{k_2} & \frac{b''_2}{k_2} \\
\end{bmatrix} * \vec{v}
\end{flalign}


\begin{flalign}
( \vec{f_3} \circ \vec{f_2} \circ \vec{f_1}) (\vec{b}) & = \begin{bmatrix}
     1 & 0 & 0 \\
     0 & \frac{b''_2}{k_2} & \frac{b''_3}{k_2} \\
     0 & \frac{-b''_3}{k_2} & \frac{b''_2}{k_2} \\
\end{bmatrix} * (\vec{f_2} \circ \vec{f_1}) (\vec{b})
\end{flalign}




\textbf{Project $( \vec{f_3} \circ \vec{f_2} \circ \vec{f_1}) (\vec{b})$ onto the $yz$ plane }

Define $\vec{f_4}(\vec{v})$ to project any vector $v$ onto the $y$.


\begin{flalign}
\vec{f_4}(\vec{v}) & \triangleq \begin{bmatrix}
     0 & 0 & 0 \\
     0 & 1 & 0 \\
     0 & 0 & 0 \\
\end{bmatrix} * \vec{v}
\end{flalign}

\begin{flalign}
( \vec{f_4} \circ \vec{f_3} \circ \vec{f_2} \circ \vec{f_1}) (\vec{b}) & = \begin{bmatrix}
     0 & 0 & 0 \\
     0 & 1 & 0 \\
     0 & 0 & 0 \\
\end{bmatrix} *  \begin{bmatrix}
     1 & 0 & 0 \\
     0 & \frac{b''_2}{k_2} & \frac{b''_3}{k_2} \\
     0 & \frac{-b''_3}{k_2} & \frac{b''_2}{k_2} \\
  \end{bmatrix} * (\vec{f_2} \circ \vec{f_1}) (\vec{b}) \\
  & = \begin{bmatrix}
     0 & 0 & 0 \\
     0 & \frac{b''_2}{k_2} & \frac{b''_3}{k_2} \\
     0 & 0 & 0 \\
\end{bmatrix} * (\vec{f_2} \circ \vec{f_1}) (\vec{b})
\end{flalign}



\textbf{Rotate $( \vec{f_4} \circ \vec{f_3} \circ \vec{f_2} \circ \vec{f_1}) (\vec{b}) $ 90 degrees on the $yz$ plane.}

Define $\vec{f_5}(\vec{v})$ rotate any vector $v$ around the $yz$ plane.

\begin{flalign}
\vec{f_5}(\vec{v}) & \triangleq \begin{bmatrix}
     1 & 0 & 0 \\
     0 & 0 & -1 \\
     0 & 1 & 0 \\
\end{bmatrix} * \vec{v}
\end{flalign}

\begin{flalign}
  ( \vec{f_5} \circ \vec{f_4} \circ \vec{f_3} \circ \vec{f_2} \circ \vec{f_1}) (\vec{b}) & = \begin{bmatrix}
     1 & 0 & 0 \\
     0 & 0 & -1 \\
     0 & 1 & 0 \\
\end{bmatrix} * \begin{bmatrix}
     0 & 0 & 0 \\
     0 & \frac{b''_2}{k_2} & \frac{b''_3}{k_2} \\
     0 & 0 & 0 \\
  \end{bmatrix} * (\vec{f_2} \circ \vec{f_1}) (\vec{b}) \\
  & = \begin{bmatrix}
     0 & 0 & 0 \\
     0 & 0 & 0 \\
     0 & \frac{b''_2}{k_2} & \frac{b''_3}{k_2} \\
  \end{bmatrix} * (\vec{f_2} \circ \vec{f_1}) (\vec{b}) \\
  & = \begin{bmatrix}
     0 & 0 & 0 \\
     0 & 0 & 0 \\
     0 & \frac{b''_2}{k_2} & \frac{b''_3}{k_2} \\
  \end{bmatrix} *  \begin{bmatrix}
    b''_1 \\
    b''_2 \\
    b''_3 \\
            \end{bmatrix} \\
  & =  \begin{bmatrix}
    0 \\
    0 \\
    \frac{{b''_2}^2 + {b''_3}^2}{\sqrt{{b''_2}^2 + {b''_3}^2}} \\
            \end{bmatrix} \\
  & =  \begin{bmatrix}
    0 \\
    0 \\
    k_2 \\
            \end{bmatrix}
\end{flalign}


\textbf{Apply inverse of $f_3$ to $( \vec{f_5} \circ \vec{f_4} \circ \vec{f_3} \circ \vec{f_2} \circ \vec{f_1}) (\vec{b})$ }

\begin{flalign}
  ( \vec{f_3}^{-1} \circ \vec{f_5} \circ \vec{f_4} \circ \vec{f_3} \circ \vec{f_2} \circ \vec{f_1}) (\vec{b}) & = \begin{bmatrix}
     1 & 0 & 0 \\
     0 & \frac{b''_2}{k_2} & \frac{-b''_3}{k_2}  \\
     0 & \frac{b''_3}{k_2} & \frac{b''_2}{k_2}  \\
\end{bmatrix} *   \begin{bmatrix}
    0 \\
    0 \\
    k_2 \\
            \end{bmatrix} \\
  & = \begin{bmatrix}
    0 \\
    -b''_3 \\
    b''_2 \\
  \end{bmatrix} \\
  & = \begin{bmatrix}
     0 \\
     \frac{a_1a_3b_1}{k_1\norm{a}} + \frac{a_2a_3b_2}{k_1\norm{a}} + \frac{-k_1b_3}{\norm{a}} \\
     \frac{-a_2b_1}{k_1}  + \frac{a_1b_2}{k_1} \\
\end{bmatrix}  \\
  & = \frac{1}{k_1\norm{a}} * \begin{bmatrix}
     0 \\
     {a_1a_3b_1} + {a_2a_3b_2} + {-k_1^2b_3} \\
     {-a_2b_1{\norm{a}}}  + {a_1b_2{\norm{a}}} \\
\end{bmatrix}
\end{flalign}



\textbf{Rotate the x axis back to $\vec{a}$ }
\begin{flalign}
  f  & \triangleq ( \vec{f_1}^{-1} \circ \vec{f_2}^{-1} \circ \vec{f_3}^{-1} \circ \vec{f_5} \circ \vec{f_4} \circ \vec{f_3} \circ \vec{f_2} \circ \vec{f_1}) (\vec{b}) \\
     & = \begin{bmatrix}
  \frac{a_1}{\norm{a}} & \frac{-a_2}{k_1} & \frac{-a_1a_3}{k_1\norm{a}}\\
  \frac{a_2}{\norm{a}} & \frac{a_1}{k_1} & \frac{-a_2a_3}{k_1\norm{a}}  \\
  \frac{a_3}{\norm{a}} & 0 & \frac{k_1}{\norm{a}} \\
\end{bmatrix} * \begin{bmatrix}
     0 \\
     \frac{a_1a_3b_1}{k_1\norm{a}} + \frac{a_2a_3b_2}{k_1\norm{a}} + \frac{-k_1b_3}{\norm{a}} \\
     \frac{-a_2b_1}{k_1}  + \frac{a_1b_2}{k_1} \\
\end{bmatrix} \\
& = \frac{1}{k_1^2\norm{a}^2} * \begin{bmatrix}
  {a_1k_1} & {-a_2\norm{a}} & {-a_1a_3} \\
  {a_2k_1} & {a_1\norm{a}} & {-a_2a_3}  \\
  {a_3k_1} & 0 & {k_1^2} \\
\end{bmatrix} * \begin{bmatrix}
     0 \\
     {a_1a_3b_1} + {a_2a_3b_2} + {-k_1^2b_3} \\
     {-a_2b_1{\norm{a}}}  + {a_1b_2{\norm{a}}} \\
\end{bmatrix} \\
& = \frac{1}{k_1^2\norm{a}^2} * \begin{bmatrix}
     \norm{a} *({-a_1a_2a_3b_1} + {-a_2^2a_3b_2} + {k_1^2a_2b_3} + {a_1a_2a_3b_1}  + {-a_1^2a_3b_2}) \\
     \norm{a} * ({a_1^2a_3b_1} + {a_1a_2a_3b_2} + {-k_1^2a_1b_3} + {a_2^2a_3b_1}  + {-a_1a_2a_3b_2}) \\
     {k_1^2\norm{a} * ({-a_2b_1}  + {a_1b_2})} \\
\end{bmatrix} \\
& = \frac{1}{k_1^2\norm{a}} * \begin{bmatrix}
     {-a_2^2a_3b_2} + {k_1^2a_2b_3}  + {-a_1^2a_3b_2} \\
     {a_1^2a_3b_1} + {-k_1^2a_1b_3} + {a_2^2a_3b_1} \\
     {k_1^2\norm{a} * ({-a_2b_1}  + {a_1b_2})} \\
\end{bmatrix} \\
& = \frac{1}{\norm{a}} * \begin{bmatrix}
     {a_2b_3} + \frac{-a_2^2a_3b_2 -a_1^2a_3b_2}{k_1^2} \\
     {-a_1b_3} + \frac{a_1^2a_3b_1 + a_2^2a_3b_1}{k_1^2} \\
     {-a_2b_1}  + {a_1b_2} \\
\end{bmatrix} \\
& = \frac{1}{\norm{a}} * \begin{bmatrix}
     {a_2b_3} + \frac{-(a_1^2 + a_2^2)a_3b_2}{k_1^2} \\
     {-a_1b_3} + \frac{(a_1^2 + a_2^2)a_3b_1}{k_1^2} \\
     {-a_2b_1}  + {a_1b_2} \\
\end{bmatrix} \\
& = \frac{1}{\norm{a}} * \begin{bmatrix}
     {a_2b_3} + \frac{-k_1^2a_3b_2}{k_1^2} \\
     {-a_1b_3} + \frac{k_1^2a_3b_1}{k_1^2} \\
     {-a_2b_1}  + {a_1b_2} \\
\end{bmatrix} \\
& = \frac{1}{\norm{a}} * \begin{bmatrix}
     {a_2b_3} - {a_3b_2} \\
     {-a_1b_3} + a_3b_1 \\
     {a_1b_2} - {a_2b_1}  \\
\end{bmatrix} \\
\end{flalign}

\textbf{Scale $\vec{f} (\vec{b})$ by $\norm{a}$}

\begin{flalign}
  \norm{a} * \vec{f} (\vec{b})
  & = \norm{a} *  \frac{1}{\norm{a}} * \begin{bmatrix}
     {a_2b_3} - {a_3b_2} \\
     {-a_1b_3} + a_3b_1 \\
     {a_1b_2} - {a_2b_1}  \\
\end{bmatrix} \\
  & =  \begin{bmatrix}
     {a_2b_3} - {a_3b_2} \\
     {-a_1b_3} + a_3b_1 \\
     {a_1b_2} - {a_2b_1}  \\
\end{bmatrix} \\
  & = a \times b
\end{flalign}
\end{proof}



\textbf{License of proof }

\copyright 2022-2023 William Emerison Six

Permission is hereby granted, free of charge, to any person obtaining a copy of this software and associated documentation files (the "Software"), to deal in the Software without restriction, including without limitation the rights to use, copy, modify, merge, publish, distribute, sublicense, and/or sell copies of the Software, and to permit persons to whom the Software is furnished to do so, subject to the following conditions:

The above copyright notice and this permission notice shall be included in all copies or substantial portions of the Software.

THE SOFTWARE IS PROVIDED "AS IS", WITHOUT WARRANTY OF ANY KIND, EXPRESS OR IMPLIED, INCLUDING BUT NOT LIMITED TO THE WARRANTIES OF MERCHANTABILITY, FITNESS FOR A PARTICULAR PURPOSE AND NONINFRINGEMENT. IN NO EVENT SHALL THE AUTHORS OR COPYRIGHT HOLDERS BE LIABLE FOR ANY CLAIM, DAMAGES OR OTHER LIABILITY, WHETHER IN AN ACTION OF CONTRACT, TORT OR OTHERWISE, ARISING FROM, OUT OF OR IN CONNECTION WITH THE SOFTWARE OR THE USE OR OTHER DEALINGS IN THE SOFTWARE.

\end{document}
